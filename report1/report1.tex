\documentclass[a4paper,11pt]{article}
\usepackage[T1]{fontenc}
\usepackage[utf8]{inputenc}
\usepackage[english]{babel}
\usepackage{lmodern}
\usepackage{amsmath}
\usepackage{url}
\usepackage{alltt}

\usepackage[top=30mm, bottom=30mm, left=25mm, right=25mm]{geometry}
\setlength\parindent{0pt}
\setlength{\parskip}{0.7em}

%\usepackage{graphicx}
%\DeclareGraphicsExtensions{.pdf,.png,.jpg}
%\graphicspath{{./images/}}

\title{INGI2325 Computer Graphics \\
  Project -- First Report}
\author{Benoit Daloze}

\begin{document}
\maketitle

\section{Project architecture}

\subsection{Main loop}

The main loop is located in \texttt{src/raytracer/RayTracer.java}.
The loop is parallelized and use all cores available.

The loop actually iterates over an iterator of $(x,y)$ coordinates and
by default draws first pixels from the center, which each thread (core) assigned an equal part (angle) of the circle around the center (thus making a nice thread visualization).

\subsection{Shapes}

The shapes are represented as a simple data class having a Material and a Geometry.

\subsection{Lights}

There is an abstract \texttt{Light} class, which requires an \emph{intensity} and a \emph{color} and ensures subclasses define \texttt{l(Point3D hit)}, which gives the $l$ normalized vector from the intersection directed to the light source used for shading.

\texttt{PointLight} is the only subclass of \texttt{Light} at this time and only has an extra attribute \texttt{position}.

The implementation supports multiple point lights and their colors, but it might result in too bright images as $c_l = 1$.

\subsection{Material}

The \texttt{Material} class is very simple and only has a (diffuse) \texttt{color} attribute.

\section{How to run and compile your project}

\begin{itemize}
  \item To compile, just run \texttt{\$ ant compile} in the top-level directory.
  \item To run the demo, type \texttt{\$ ant} in the top-level directory.
  \item To remove compiled files, run \texttt{\$ ant clean}.
\end{itemize}

Of course, the project can be imported in Eclipse and ran from there.

It also has a command-line interface:
\begin{verbatim}
$ java -cp bin raytracer.CLI <SCENE>.sdl <OUTPUT>.png
\end{verbatim}

\section{Other features}

Produced images from the examples (by the script \texttt{run.rb}) are in \texttt{output/}.
The script needs \texttt{.obj} and \texttt{.obj.gz} files to be in \texttt{XML/}.

The \texttt{.obj.gz} files are decompressed on the fly by the \texttt{FileGeometryParser}, thus not needing explicit decompression.

\end{document}
