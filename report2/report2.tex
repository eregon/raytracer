\documentclass[a4paper,11pt]{article}
\usepackage[T1]{fontenc}
\usepackage[utf8]{inputenc}
\usepackage[english]{babel}
\usepackage{lmodern}
\usepackage{amsmath}
\usepackage{url}
\usepackage{alltt}

\usepackage[top=30mm, bottom=30mm, left=25mm, right=25mm]{geometry}
\setlength\parindent{0pt}
\setlength{\parskip}{0.7em}

%\usepackage{graphicx}
%\DeclareGraphicsExtensions{.pdf,.png,.jpg}
%\graphicspath{{./images/}}

\title{INGI2325 Computer Graphics \\
  Project -- Last Report}
\author{Benoit Daloze}

\begin{document}
\maketitle

\section{Features}

\begin{itemize}
  \item Bounding Volume Hierarchy
  \item Further geometrical primitives: sphere, circle, cube
  \item Antialiasing via Supersampling
  \item Area lights and soft shadows
  \item Moving and re-rendering the scene
\end{itemize}

Use the arrow keys to move the camera, \emph{P} to zoom, \emph{M} to unzoom, and \emph{Q, S, D, Z} to translate the camera.

\section{How to run and compile your project}

\begin{itemize}
  \item To compile, just run \texttt{\$ ant compile} in the top-level directory.
  \item To run the demo, type \texttt{\$ ant} in the top-level directory.
  \item To remove compiled files, run \texttt{\$ ant clean}.
\end{itemize}

Of course, the project can be imported in Eclipse and ran from there.

It also has a command-line interface:
\begin{verbatim}
$ java -cp bin raytracer.CLI [OPTIONS] <SCENE>.sdl <OUTPUT>.png
  -gui          Render in a window
  -width=N
  -height=N
  -noShadows
  -ss=N         Supersampling (N*N samples per pixel)
  -bb           Draw bounding boxes
\end{verbatim}

\end{document}
